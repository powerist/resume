%%%%%%%%%%%%%%%%%%%%%%%%%%%%%%%%%%%%%%%%%
% Long Sectioned Curriculum Vitae
% LaTeX Template
% Version 1.1 (9/12/12)
%
% This template has been downloaded from:
% http://www.latextemplates.com
%
% Original author:
% Rensselaer Polytechnic Institute (http://www.rpi.edu/dept/arc/training/latex/resumes/)
%
% Important note:
% This template requires the res.cls file to be in the same directory as the
% .tex file. The res.cls file provides the resume style used for structuring the
% document.
%
%%%%%%%%%%%%%%%%%%%%%%%%%%%%%%%%%%%%%%%%%

%----------------------------------------------------------------------------------------
%	PACKAGES AND OTHER DOCUMENT CONFIGURATIONS
%----------------------------------------------------------------------------------------

\documentclass[10pt]{res} % Use the res.cls style, the font size can be changed to 11pt or 12pt here

\usepackage{helvet} % Default font is the helvetica postscript font
%\usepackage{newcent} % To change the default font to the new century schoolbook postscript font uncomment this line and comment the one above
\usepackage{hyperref}

\newsectionwidth{0pt} % Stops section indenting

\long\def\symbolfootnote[#1]#2{\begingroup%
\def\thefootnote{\fnsymbol{footnote}}\footnote[#1]{#2}\endgroup}

\begin{document}

%----------------------------------------------------------------------------------------
%	YOUR NAME AND ADDRESS(ES) SECTION
%----------------------------------------------------------------------------------------

\name{{\bf Chen Chen}\\ \\} % Your name at the top

% If you don't want one of the addresses, simply remove all the text in the first or second \address{} bracket

\address{University of Pennsylvania \\ Room 575, Levine Hall \\ 3330 Walnut Street \\ Philadelphia, PA 19104-6309} % Your address 1

\address{Email: chenche@seas.upenn.edu \\ 
Web:  \url{http://www.cis.upenn.edu/~chenche/} \\ 
Phone: (215) 290-9716}
%Office: (215) 573-2582} % Your address 2

%----------------------------------------------------------------------------------------

\begin{resume}

%----------------------------------------------------------------------------------------
%	OBJECTIVE SECTION
%----------------------------------------------------------------------------------------

%\section{\centerline{OBJECTIVE}}

%\vspace{8pt} % Gap between title and text

%A position in Personnel Administration utilizing skills in recruiting, training and compensation.\\

%----------------------------------------------------------------------------------------
%	EDUCATION SECTION
%----------------------------------------------------------------------------------------

\section{\centerline{EDUCATION}}

\vspace{15pt} % Gap between title and text

\begin{itemize}
\item {\bf Ph.D. Candidate},
Computer and Information Science \hfill Sep 2011 - Present\\
University of Pennsylvania, Philadelphia, PA  \\
Advisors: Boon Thau Loo and Limin Jia (CMU) \\
GPA: 3.93/4.00

\item {\bf Bachelor of Science}, Information Security \hfill Sep 2007 - Sep 2011\\
School of Computer Science \\
Fudan University, Shanghai, China  \\
GPA: 3.56/4.00; Major GPA: 3.90/4.00; Rank: 3/32
\end{itemize}

%----------------------------------------------------------------------------------------

 \vspace{0.2in} % Some whitespace between sections

%----------------------------------------------------------------------------------------
%	PUBLICATIONS SECTION
%----------------------------------------------------------------------------------------

\section{\centerline{PUBLICATIONS}}

\vspace{15pt} % Gap between title and text
\begin{itemize} \itemsep -2pt % Reduce space between items
\item {\bf Automated Verification of Safety Properties in Declarative Networking
    Programs} \\
Chen Chen, Lay Kuan Loh, Limin Jia, Wenchao Zhou and Boon Thau Loo\\
In the 17th International Symposium on Principles and Practice of Declarative
Programming, July 2015\\
\item {\bf A Scalable Multi-Datacenter Layer-2 Network Architecture} \\
Chen Chen, Changbin Liu, Pingkai Liu, Boon Thau Loo and Ling Ding\\
In ACM Sigcomm Symposium on SDN Research (SOSR), June 2015\\
\item {\bf A Program Logic for Secure Routing Protocols} \\
Chen Chen, Limin Jia, Hao Xu, Cheng Luo, Wenchao Zhou and Boon Thau Loo\\
In the 34th IFIP International Conference on Formal Techniques for Distributed
Objects, Components and Systems (FORTE 2014), June 2014\\
\item {\bf Proof-based Verification of Software Defined Networks} \\
Chen Chen, Limin Jia, Wenchao Zhou, and Boon Thau Loo \\
In the Open Networking Summit (ONS), March 2014\\
\item {\bf Reduction-based Security Analysis of Internet Routing Protocols} \\
Chen Chen, Limin Jia, Wenchao Zhou, and Boon Thau Loo \\
In the 2nd International Workshop on Rigorous Protocol Engineering (WRiPE), Oct 2012 \\
\item {\bf Datacast: A Scalable and Efficient Reliable Group Data Delivery Service for Data Centers} \\
Jiaxin Cao, Chuanxiong Guo, Guohan Lu, Yongqiang Xiong, Yixin Zheng, Yongguang
Zhang, Yibo Zhu, and Chen Chen \\
In the 8th International Conference on emerging Networking EXperiments and Technologies (CoNEXT)\\
\item {\bf Towards a Secure and Verifiable Future Internet} \\
Limin Jia, Chen Chen, Sangeetha A.Jyothi, Wenchao Zhou, Suyog Mapara and Boon Thau Loo \\
In the Off the Beaten Track: Underrepresented Problems for Programming Language Researchers(OBT)\\
\end{itemize}

%----------------------------------------------------------------------------------------

\vspace{0.2in} % Some whitespace between sections
\vspace{0.8in} % Some whitespace between sections

%----------------------------------------------------------------------------------------
%	PROFESSIONAL EXPERIENCE SECTION
%----------------------------------------------------------------------------------------

\section{\centerline{RESEARCH EXPERIENCE}}

\vspace{8pt} % Gap between title and text

\subsubsection{University of Pennsylvania, Research Assistant, Sep 2011 - Present}
\begin{itemize}
\item {\bf Distributed Provenance Compression} \hfill Sep 2015 -- Jun
  2016\\ Designed and implemented a framework for compressing network
  provenance in a distributed fashion. The framework places provenance
  trees into different equivalence classes, and stores only one
  concrete copy for provenance trees in the same equivalence class. To
  efficiently identify the equivalence class to which each provenance tree
  belongs, we performed static analysis on provenance maintenance
  sourcecode, and used the analysis results to compress provenance
  trees at runtime.


(in collaboration with Prof. Boon Thau Loo, Prof. Limin Jia and Prof. Wenchao Zhou)
\item {\bf Verification of Declarative Networking Programs} \hfill Jun
  2014 -- May 2015\\ Designed and implemented a framework for
  verifying safety properties of networking programs specified in
  declarative programming languages (e.g. NDLog). The declarative
  specification of networking applications is parsed and converted
  into dependency graph -- a data structure capturing the tuple-level
  dependency between relations in NDLog. To verify The safety
  properties -- which are speicifed in a restricted form of
  first-order logic -- we used the dependency graph to generate all
  candidate system execution traces, and ensure that the specified
  property holds on all traces. We developed a prototype to verify
  properties of SDN applications, and found several bugs in the
  applications.


(in collaboration with Prof. Boon Thau Loo, Prof. Limin Jia and Prof. Wenchao Zhou)
\item {\bf Program Logic for Secure Routing Protocols} \hfill Sep 2011
  -- May 2014\\ Developed a program logic for verifying security
  properties of secure routing protocols (e.g., Secure BGP). The logic
  is built on SANDlog, a variant specification language of Network
  Datalog (i.e., NDLog) with extension of security primitives, We used
  the logic to prove classical properties, such as path authenticity,
  of secure routing protocols. Our logic is proved to be sound with
  regards to the semantics of SANDlog.


(in collaboration with Prof. Boon Thau Loo, Prof. Limin Jia and Prof. Wenchao Zhou)
\end{itemize}

\subsubsection{NEC Labs America, Research Assistant, Jun 2015 - Aug 2015}
\begin{itemize}
\item {\bf Data Analytics for Networked System Logs} \hfill Jun 2015 - Aug 2015\\
Developed a knowledge management framework for log analytics in networked
systems. The framework allows the user to specify desired network behaviors
(e.g., TCP setup), and identifies the sequence of log entries that match the
specified behavior. The framework also supports partial matching when the log
information is incomplete.

(in collaboration with Dr. Hui Zhang, Dr. Qiang Xu, and Dr. Biplob Debnath)
\end{itemize}


\subsubsection{AT\&T Labs Research, Research Assistant, Jun 2013 - Aug 2013}
\begin{itemize}
\item {\bf Software Defined Networks} \hfill Jun 2013 - Aug 2013\\
Built infrastructure supporting multi-datacenter Layer-2 Network. SDN controllers
are used to achieve scalability by replacing broadcast traffic, such as ARP
requests and DHCP requests/replies, with unicast traffic to the controller. Live
migration can be supported naturally by our system.

(in collaboration with Dr. Changbin Liu (Senior Member of Technical Staff, AT\&T) and Prof. Boon Thau Loo)
\end{itemize}

\newpage

\subsubsection{Microsoft Research Asia, Research Assistant, Apr 2010 - Oct 2010}
\begin{itemize}
\item {\bf Multicast in Data Center Network} \hfill Apr 2010 - Oct 2010\\
Developed algorithms for multicast in data center network. Constructed optimal
multicast trees for different large-scale architectures in seconds, including
Bcube, FatTree and Torus. Compared algorithms with BitTorrent protocols for
optimization evaluation.

(in collaboration with Dr. Jiaxin Cao, Dr. Chuanxiong Guo, Dr. Haitao Wu, Dr. Yongqiang Xiong)
\end{itemize}

\subsubsection{Fudan University, Research Assistant, Sep 2009 - Feb 2010}
\begin{itemize}
\item {\bf Network Coding for Highly Reliable P2P Network} \hfill Sep 2009 - Feb 2010\\
Implemented hierarchical P2P network based on network coding with C$\#$. Reduced
file sharing time in distributed file transmission with network coding. Compared
the performance of network coding with traditional file sharing protocol:
BitTorrent and showed that network coding is quicker and effective in
distributing scarce information.

(in collaboration with Prof. Xin Wang, Prof. Jin Zhao)
\end{itemize}

%----------------------------------------------------------------------------------------

%\vspace{0.2in} % Some whitespace between sections

%----------------------------------------------------------------------------------------
%	COMPUTER SKILLS SECTION
%----------------------------------------------------------------------------------------

%\section{\centerline{COMPUTER SKILLS}}

%\vspace{8pt} % Gap between title and text

%Experienced in SPSS, BMDP statistical packages; LOTUS 1-2-3, Personal Editor

%----------------------------------------------------------------------------------------


\vspace{0.2in} % Some whitespace between sections

%----------------------------------------------------------------------------------------
%	TEACH ASSISTANT SECTION
%----------------------------------------------------------------------------------------

% \section{\centerline{TEACHING EXPERIENCE}}

% \vspace{15pt} % Reduce space between section title and contents

% \begin{itemize}
% \item CIS 262: Automata, Computability and Complexity \hfill Fall 2012\\
% Teaching assistant for Prof. Aaron Roth, responsible for recitation
% \item CIS 511: Theory of Computation \hfill Spring 2013 \\
% Teaching Assistant for Prof. Jean Gallier
% \end{itemize}
%----------------------------------------------------------------------------------------

% \vspace{0.2in} % Some whitespace between sections

%----------------------------------------------------------------------------------------
%	HONORS SECTION
%----------------------------------------------------------------------------------------

\section{\centerline{HONORS}}

\vspace{15pt} % Reduce space between section title and contents

\begin{itemize}
\item {\bf National Scholarship 1st Prize} \hfill 2009 - 2010\\
Top 1$\%$ student in Fudan University(i.e. Top 1 student in the major) is awarded for excellent academic performance. One student every two years for the same major
\item {\bf Tung OOCL Scholarship 1st Prize} \hfill 2008 - 2009 \\
Top 1$\%$ student in the major is awarded for excellent academic performance
\end{itemize}

%----------------------------------------------------------------------------------------

\vspace{0.2in} % Some whitespace between sections

%----------------------------------------------------------------------------------------
%	COURSES SECTION
%----------------------------------------------------------------------------------------

% \section{\centerline{COURSES TAKEN}}

% \vspace{15pt} % Reduce space between section title and contents

% \begin{itemize}
% \item CIS 673: Computer-Aided Verification \hfill Fall 2012
% \item CIS 700: Programming and Problem Solving \hfill Summer 2012
% \item CIS 500: Software Foundation \hfill Spring 2012
% \item CIS 511: Theory of Computation \hfill Spring 2012
% \item CIS 501: Introduction to Computer Architecture \hfill Fall 2011
% \item CIS 502: Analysis of Algorithm \hfill Fall 2011
% \item CIS 800: Rigorous Internet Protocol Engineering \hfill Fall 2011
% \end{itemize}

%----------------------------------------------------------------------------------------

\end{resume}
\end{document} 
