\subsubsection{NEC Labs America, Research Assistant, Jun 2015 - Aug 2015}
\begin{itemize}
\item {\bf Data Analytics for Networked System Logs} \hfill Jun 2015 - Aug 2015\\
Developed a knowledge management framework for log analytics in networked
systems. The framework allows the user to specify desired network behaviors
(e.g., TCP setup), and identifies the sequence of log entries that match the
specified behavior. The framework also supports partial matching when the log
information is incomplete.

(in collaboration with Dr. Hui Zhang, Dr. Qiang Xu, and Dr. Biplob Debnath)
\end{itemize}


\subsubsection{AT\&T Labs Research, Research Assistant, Jun 2013 - Aug 2013}
\begin{itemize}
\item {\bf Software Defined Networks} \hfill Jun 2013 - Aug 2013\\
Built infrastructure supporting multi-datacenter Layer-2 Network. SDN controllers
are used to achieve scalability by replacing broadcast traffic, such as ARP
requests and DHCP requests/replies, with unicast traffic to the controller. Live
migration can be supported naturally by our system.

(in collaboration with Dr. Changbin Liu (Senior Member of Technical Staff, AT\&T) and Prof. Boon Thau Loo)
\end{itemize}


\subsubsection{Microsoft Research Asia, Research Assistant, Apr 2010 - Oct 2010}
\begin{itemize}
\item {\bf Multicast in Data Center Network} \hfill Apr 2010 - Oct 2010\\
Developed algorithms for multicast in data center network. Constructed optimal
multicast trees for different large-scale architectures in seconds, including
Bcube, FatTree and Torus. Compared algorithms with BitTorrent protocols for
optimization evaluation.

(in collaboration with Dr. Jiaxin Cao, Dr. Chuanxiong Guo, Dr. Haitao Wu, Dr. Yongqiang Xiong)
\end{itemize}

\subsubsection{Fudan University, Research Assistant, Sep 2009 - Feb 2010}
\begin{itemize}
\item {\bf Network Coding for Highly Reliable P2P Network} \hfill Sep 2009 - Feb 2010\\
Implemented hierarchical P2P network based on network coding with C$\#$. Reduced
file sharing time in distributed file transmission with network coding. Compared
the performance of network coding with traditional file sharing protocol:
BitTorrent and showed that network coding is quicker and effective in
distributing scarce information.

(in collaboration with Prof. Xin Wang, Prof. Jin Zhao)
\end{itemize}

%----------------------------------------------------------------------------------------

\vspace{0.2in} % Some whitespace between sections
